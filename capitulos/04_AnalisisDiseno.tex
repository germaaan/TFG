\chapter{Análisis y diseño}

\section{Análisis de requisitos}

Todo software se desarrolla para cubrir una necesidad, por lo que en este apartado vamos a describir los requisitos que se 
estiman necesarios para cubrir los objetivos propuestos.

\subsection{Descripción de los actores}

El usuario de la aplicación será cualquier persona que tenga interés por conocer datos internos de la Universidad de Granada 
fácilmente. Como no se quiere enfocar en un público objetivo, el portal tiene que ser fácil de utilizar tanto para personas 
con experiencia en la navegación de páginas web como para las que no la tengan.

\bigskip
También tenemos al desarrollador, que será el encargado de introducir los elementos de información que serán listado en la
plataforma. Como es una página estática, los elementos serán mostados en la tabla correspondiente de la página correspondiente
después de que sean leidos del archivo JSON correspondiente.

\subsection{Requisitos Funcionales}

Los requisitos funcionales son las características que tiene que implementar el sistema para cubrir todas las necesidades de 
los distintos usuarios. Al usuario lo único que le interesa es ver una página web estática con la información que desea 
consultar, por lo que el único requisito imprescindible es que cuando pulse una categoría esta se despliegue y cuando se 
pulse un enlace a OpenData UGR este nos lleve al conjunto de datos correspondiente. Además, se habilitará un buscador para que
también se pueda acceder a la información mediante la introducción de palabras clave. 

\newpage
Todos los requisitos funcionales que tratará este proyecto están únicamente dirigidos a la plataforma UGR Transparente.

\begin{itemize}
  \item \textbf{RF-1.} Introducción de nuevos elementos de información.
\end{itemize}

\begin{itemize}
  \item \textbf{RF-2.} Acceso a la información:
  \begin{itemize}
    \item \textbf{RF-2.1.} Al seleccionar una categoría esta se despliega y aparecen sus subcategorías.
    \item \textbf{RF-2.2.} Al seleccionar una subcategoría se muestran todos sus elementos.
    \item \textbf{RF-2.2.} Cuando se pulsa sobre el enlace de un elemento, automáticamente se accede a la información contenida 
    de ese elemento en OpenData UGR.
    \item \textbf{RF-2.3.} Para obtener información sobre un tema en concreto se introducirá el termino clave relacionado en 
    el buscador.
    \end{itemize}
\end{itemize}

\begin{itemize}
  \item \textbf{RF-3.} Presentación de la información:
  \begin{itemize}
    \item \textbf{RF-.1.} Generar tablas con los elementos de información.
    \item \textbf{RF-3.1.} Si el elemento de información no es un archivo PDF este se podrá previsualizar desde el mismo portal
    de transparencia sin necesidad de acceder a OpenData UGR.
    \item \textbf{RF-3.2.} Siempre se podrá descargar el archivo con la información contenido en OpenData UGR sobre cualquier 
    elemento listado.
  \end{itemize}
\end{itemize}
	
Los aspectos de funcionalidad ya se encuentran implementados en su mayoría (solo existen problemas al cargar la información) de
una fase previa al proyecto por lo que esta será la base de la que se partirá para el desarrollo.

\subsection{Requisitos no Funcionales}

Los requisitos no funcionales son las características propias del desarrollo, pero que no tienen que estar relacionadas con su 
funcionalidad. En este caso nos referimos a todas las características que se requieren para que la aplicación siga un 
desarrollo ágil de despliegue continuo y administración automatica de la plataforma UGR Transparente.

\begin{itemize}
  \item \textbf{RN-1.} Procesar la información que será mostrada en las distintas páginas del portal.
  \item \textbf{RN-2.} Implementar tests unitarios para validar que todas las funcionalidades programadas funcionan 
  correctamente.
  \item \textbf{RN-3.} Realizar test de cobertura que compruebe la fiabilidad que proporcionan los tests unitarios.
  \item \textbf{RN-4.} Usar integración continua para asegurarse que los cambios introducidos no producen conflictos en la 
  plataforma.
  \item \textbf{RN-5.} Usar despliegue automático para actualizar con los nuevos cambios la plataforma una vez estos han sido 
  validados.
  \item \textbf{RN-6.} Elaborar una aprovisionamiento que permita que en una infraestructura por determinar se pueda instalar 
  automáticamente la plataforma y todos los elementos necesarios.
\end{itemize}

En cuanto a la plataforma OpenData UGR, este proyecto solo contempla un único objetivo no funcional que está relacionado con
la administración del sistema más que con que aspectos de desarrollo software.

\begin{itemize}
  \item \textbf{RN-7.} Crear una aprovisionamiento y personalización de CKAN para OpenData UGR.
\end{itemize}

\subsection{Requisitos de Información}

Los requisitos de información se refieren a la información que es necesaria almacenar en el sistema. La única información 
relevante que se va a almacenar son los datos descriptivos y de enlace de cada uno de los elementos del portal OpenData UGR 
que se van a mostrar en UGR Transparente.

\begin{itemize}
  \item \textbf{RI-1.} Datos abiertos.
  \begin{itemize}
    \item Información sobre cada uno de los elementos que se van a mostrar en el portal de transparencia como datos abiertos.
    \item Contenido: nombre, categoría, conjunto de datos, enlace a OpenData UGR, enlace al recurso.
  \end{itemize}
\end{itemize}

\section{Modelos de casos de uso}

Aunque ya se ha indicado que la parte funcional ya se encuentra implementada de forma previa a este proyecto, se van a incluir
unos modelos de caso de uso simples para dar un visión más clara del funcionamiento general de la plataforma UGR Transparente.

\newpage
\subsection{Descripción básica de actores}

\begin{itemize}
  \item \textbf{Ac-1.} Usuario
  \begin{itemize}
   \item Descripción: Persona que usa la plataforma que consulta datos.
   \item Características: Es el usuario común que accederá a la página.
   \item Relaciones: Ninguna.
   \item Atributos: Ninguno.
   \item Comentarios: El usuario no es necesario que tenga ningún conocimiento previo al uso de la plataforma, simplemente
   accederá y consultará los datos que sean de su interes.
  \end{itemize}
  
  \item \textbf{Ac-1.} Desarrollador
  \begin{itemize}
   \item Descripción: Encargado de añadir los elementos de información a la plataforma.
   \item Características: Su trabajo está en el lado del servidor que genera la página, nunca trabaja desde el lado del cliente.
   \item Relaciones: Ninguna.
   \item Atributos: Ninguno.
   \item Comentarios: Es el encargado de desarrollar las funcionalidades del portal, entre ellas añadir nuevos elementos de
   información.
  \end{itemize}
\end{itemize}

\subsection{Descripción casos de uso}

\begin{itemize}
 \item \textbf{CU-1.} Añadir elemento
 \begin{itemize}
  \item Actores: Desarrollador
  \item Tipo: Primario, Esencial
  \item Precondición: Ninguna
  \item Postcondición: Nuevos elementos de información serán visualizados en el portal de transparencia
  \item Propósito: Son añadidos nuevos elementos de información al portal de transparencia.
  \item Resumen: Aparecerán nuevos elementos de informaciónen la categória que corresponda, acompañado además de su enlace
  correspondiente al OpenData UGR, un botón para previsualizar la información y otro botón para descargar el elemento.
 \end{itemize}
\end{itemize}

\newpage
\begin{itemize}
 \item \textbf{CU-2.} Consultar elementos
 \begin{itemize}
  \item Actores: Usuario
  \item Tipo: Primario, esencial
  \item Precondición: Existan elementos de información.
  \item Postcondición: Se muestran los elementos de información de un determinado conjunto de datos.
  \item Propósito: Obtiene los elementos de información de un determinado conjunto de datos.
  \item Resumen: Accediendo a través del panel principal o el buscador se obtiene una tabla con los elementos de información
  de un determinado conjunto de datos.
 \end{itemize}
\end{itemize}

\begin{itemize}
 \item \textbf{CU-3.} Acceder enlace de elemento
 \begin{itemize}
  \item Actores: Usuario
  \item Tipo: Primario, esencial
  \item Precondición: Se hayan generado las tablas con los elementos de información.
  \item Postcondición: 
  \item Propósito: Accede a la información del elemento contenida en OpenData UGR.
  \item Resumen: Cuando se pulsa el enlace, se accede al conjunto de datos que contiene la información del elemento en OpenData
  UGR, presentándolo de distinta forma en función del formato del elemento y dando la opción de descagar ese mismo elemento en 
  un archivo con su formato original.
 \end{itemize}
\end{itemize}

\begin{itemize}
 \item \textbf{CU-4.} Previsualizar elemento de información
 \begin{itemize}
  \item Actores: Usuario
  \item Tipo: Secundario
  \item Precondición: Se hayan generado las tablas con los elementos de información.
  \item Postcondición: 
  \item Propósito: Previsualiza los datos del elemento.
  \item Resumen: Cuando se selecciona el botón de ver un elemento de la tabla se abre una ventana emergente en la que se 
  muestra la información del elemento que se contiene en OpenData UGR.
 \end{itemize}
\end{itemize}

\newpage
\begin{itemize}
 \item \textbf{CU-5.} Descargar elemento de información
 \begin{itemize}
  \item Actores: Usuario
  \item Tipo: Secundario
  \item Precondición: Se hayan generado las tablas con los elementos de información.
  \item Postcondición: 
  \item Propósito: Descagar el elemento en un archivo con su formato original.
  \item Resumen: Cuando se selecciona el botón de descargar un elemento de la tabla se abre una ventana emergente en la que se
  muestra la información del elemento que se contiene en OpenData UGR.
 \end{itemize}
\end{itemize}

\subsection{Diagramas de casos de uso}

Desde un punto de vista funcional es el usuario el que realiza la mayoria de las acciones del portal, la función del 
desarrollar aunque fundamental, solo es una acción que se realiza en un segundo plano del que no se tiene constancia desde 
fuera.

\begin{figure}[!h]
  \begin{center}
  \includegraphics[width=1\textwidth]{imagenes/diagrama_casos_uso.png}
  \caption[Casos de uso]{Diagrama de casos de uso}
  \label{fig:casosUso}
  \end{center}
\end{figure}

