\chapter{Introducción}

En la actualidad, nuestro país al igual que muchos otros se rige por lo que se conoce como \textbf{``democracia''}. Si buscamos
el significado de esta palabra, encontraremos que su definición es: 

\begin{quote}Sistema político que defiende la soberanía del pueblo y el derecho del pueblo a elegir y controlar a sus 
gobernantes.
\newline(\url{http://www.oxforddictionaries.com/es/definicion/espanol/democracia})
\end{quote}

Aunque \textit{``el derecho del pueblo a elegir''} es lo importante, para que esa elección pueda ser coherente, primero será el propio
pueblo el que deberá tener la obligación a conocer lo que elige, y la única forma de conocer lo que se elige es a través de la
transparencia. Por este motivo, el pueblo debe exigir transparencia en el funcionamiento de las instituciones públicas, y esas
instituciones públicas siempre no tengan nada que esconder debería facilitar de buena fe (y no porque haya una ley que les 
obligue) todo la información que el ciudadano le requiera.

\bigskip
Estas inquietudes fueron las que hicieron en un primer momento que se comenzara con el desarrollo del Portal de Transparencia
de la Universidad de Granada a principios del año pasado, cuyo desarrollo le fue encargado a la Oficina de Software Libre de la propia universidad y 
aproximadamente a los 7 meses de comenzar su desarrollo fue presentada una primera versión.

\bigskip
En febrero de este año entré como becario en la Oficina de Software Libre y me integré en el equipo de desarrollo del portal,
siendo mis mayores contribuciones cambiar el origen de datos de la página para eliminar un error que se producia recurrentemente
durante la navegación por el parte, y además, implantar una metodología de desarrollo DevOps.

\newpage
Una metodología de desarrollo DevOps consiste inicialmente en no hacer distinción entre el desarrollo del software y la
administración del mismo, todo estará comunicado para que sea posible realizar entregas del software de forma frecuente 
asegurándose de que esas mismas entregas continuas no sea el origen de fallos futuros. La forma de asegurarse de que esos
fallos no se producirán es dividir todo el desarrollo en fases que tengan que realizarse secuencialmente, controlando a cada
fase que no se produzcan errores en la misma; como una cadena de montaje en la que podemos estar seguro de que el producto
que llega al final está en perfectas condiciones, porque en caso contrario hubiera sido retirado durante el proceso.

\bigskip
En este proyecto, hemos considerado que las fases que nos permitirían asegurarnos que el producto que sale de la cadena de 
montaje en perfectas condiciones sean los test unitarios, la integración continua y el despliegue automático, todo esto apoyado
en un sistema de control de versiones y una plataforma de desarrollo colaborativo abierto. Además, también se usará
aprovisionamiento para facilitar la portabilidad del portal de una infraestructura a otra distinta. Se ahondará en el 
funcionamiento de todos estos conceptos en el capítulo 5 (Implementación), donde se explicará detenidamente como han sido 
integrados en el proyecto.