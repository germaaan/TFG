\addcontentsline{toc}{chapter}{Anexo. Realización del trabajo de fin de grado con licencia libre y formato abierto}
\chapter*{Anexo. Realización del trabajo de fin de grado con licencia libre y formato abierto}

Cuando empecé a trabajar en el proyecto ya había un trabajo previo, entre sus particularidades estaba que era un proyecto cuyo software estaba siendo directa y totalmente liberado en {\tt GitHub} bajo una licencia libre (concretamente \textbf{GNU General Public License v3}) según iba siendo desarrollado.

\bigskip
Partiendo de esta base mi tutor me propuso que realizara toda la documentación del proyecto bajo esta misma metodología. Al principio se me hacía raro concebir el desarrollo de un documento en un sitio generalmente orientado al desarrollo de software, aunque seguramente el sistema de control de versiones sería algo a lo que podría sacarle mucha ventaja.

\bigskip
Ya que el proyecto se estaba desarrollando en una licencia libre decidí que la documentación debía seguir también totalmente ese camino, por lo que el uso de una licencia \textbf{Creative Commons} era la que mejor se adaptaba a mi objetivo, además, permite que este trabajo fluya en ese mismo camino del software libre: puedes leerlo, puedes copiarlo, puedes modificarlo y finalmente, puedes seguir distribuyéndolo; lo único que tienes que hacer es reconocer mi autoría original.

\bigskip
El siguiente paso natural en la evolución de este proyecto es que la documentación tenía también que ser realizada en un formato abierto, y hablando de documentación, el formato técnico por excelencia es \LaTeX. Este sistema de composición de textos puede ser bastante complicado de usar al principio, sin embargo una vez acostumbrado llegado a ser mucho más cómodo de usar que los típicos procesadores de texto, sobretodo debido a que no es necesario preocuparse de revisar continuamente que los estilos o el formato se han estropeado; una vez que los estilos y los formatos están definidos tenemos la seguridad de estos no se van a ``estropear'', por lo que solo deberemos centrarnos en el contenido del documento que es lo realmente importante.

\bigskip
Según iba desarrollando la documentación me acostumbré a plantear las tareas y objetivos como si de un desarrollo de software cualquier se tratara, así podría aprovechar las ventajas que me proporcionaba {\tt GitHub} en este ámbito. Organizarme las tareas a base de \textit{``issues''} a los que le asignaba una etiqueta según la finalidad de la misma, pudiendo diferenciar entre tareas para añadir un nuevo contenido o corregir contenido en el que hubiera algún error. Este mismo sistema también me servía para estar en contacto con mi tutor y que pudiera seguir el desarrollo del mismo, pudiendo comentarme en cualquier momento las observaciones que considerara pertinentes, pero no solo él, al ser una plataforma pública cualquier persona podría interesada podría hacer comentarios o sugerencias para mejorar el proyecto, haciendo que aumente considerablemente la riqueza de contenidos que a este proyecto podrían ser añadidos.

\bigskip
Si hubiera elegido seguir una metodología más tradicional para desarrollar este documento, en diversos momentos su avance me hubiera sido más rápido y fácil, pero sin lugar a dudas no hubiera sido una experiencia tan enriquecedora en la que habría aprendido tanto como he aprendido siguiendo esta forma de trabajar.